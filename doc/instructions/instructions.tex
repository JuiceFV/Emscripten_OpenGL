\documentclass[12pt]{article}
\usepackage[T2A]{fontenc}
\usepackage[russian, english]{babel}
\usepackage[utf8x]{inputenc}
\usepackage{amsmath, amssymb}
\usepackage{graphicx}
\usepackage{physics}
\usepackage{float}
\usepackage{booktabs} 
\usepackage{amsmath}
\usepackage{systeme}
\usepackage{blindtext}
\usepackage{enumitem}
\usepackage{listings}
\usepackage[table]{xcolor}
\usepackage{hyperref}
\usepackage{tikz}
\usepackage{titlesec}
\usepackage{mathtools}
\usepackage{tcolorbox}
\usepackage{bm}
\usepackage[normalem]{ulem}
\usepackage{geometry}
\usepackage{siunitx}
\usepackage{xcolor}
\usepackage{tikzsymbols}
\usepackage{relsize}
\usepackage{pgfplots}
\usepackage[edges]{forest}

\definecolor{folderbg}{RGB}{124,166,198}
\definecolor{folderborder}{RGB}{110,144,169}
\newlength\Size
\setlength\Size{4pt}

\newcommand\YAMLcolonstyle{\color{red}\mdseries}
\newcommand\YAMLkeystyle{\color{black}\bfseries}
\newcommand\YAMLvaluestyle{\color{blue}\mdseries}

\makeatletter

% here is a macro expanding to the name of the language
% (handy if you decide to change it further down the road)
\newcommand\language@yaml{yaml}

\expandafter\expandafter\expandafter\lstdefinelanguage
\expandafter{\language@yaml}
{
  keywords={true,false,null,y,n},
  keywordstyle=\color{darkgray}\bfseries,
  basicstyle=\YAMLkeystyle,                                 % assuming a key comes first
  sensitive=false,
  comment=[l]{\#},
  morecomment=[s]{/*}{*/},
  commentstyle=\color{purple}\ttfamily,
  stringstyle=\YAMLvaluestyle\ttfamily,
  moredelim=[l][\color{orange}]{\&},
  moredelim=[l][\color{magenta}]{*},
  moredelim=**[il][\YAMLcolonstyle{:}\YAMLvaluestyle]{:},   % switch to value style at :
  morestring=[b]',
  morestring=[b]",
  literate =    {---}{{\ProcessThreeDashes}}3
                {>}{{\textcolor{red}\textgreater}}1     
                {|}{{\textcolor{red}\textbar}}1 
                {\ -\ }{{\mdseries\ -\ }}3,
}

% switch to key style at EOL
\lst@AddToHook{EveryLine}{\ifx\lst@language\language@yaml\YAMLkeystyle\fi}
\makeatother

\newcommand\ProcessThreeDashes{\llap{\color{cyan}\mdseries-{-}-}}

\tikzset{%
  folder/.pic={%
    \filldraw [draw=folderborder, top color=folderbg!50, bottom color=folderbg] (-1.05*\Size,0.2\Size+5pt) rectangle ++(.75*\Size,-0.2\Size-5pt);
    \filldraw [draw=folderborder, top color=folderbg!50, bottom color=folderbg] (-1.15*\Size,-\Size) rectangle (1.15*\Size,\Size);
  },
  file/.pic={%
    \filldraw [draw=folderborder, top color=folderbg!5, bottom color=folderbg!10] (-\Size,.4*\Size+5pt) coordinate (a) |- (\Size,-1.2*\Size) coordinate (b) -- ++(0,1.6*\Size) coordinate (c) -- ++(-5pt,5pt) coordinate (d) -- cycle (d) |- (c) ;
  },
}
\forestset{%
  declare autowrapped toks={pic me}{},
  declare boolean register={pic root},
  pic root=0,
  pic dir tree/.style={%
    for tree={%
      folder,
      font=\ttfamily,
      grow'=0,
    },
    before typesetting nodes={%
      for tree={%
        edge label+/.option={pic me},
      },
      if pic root={
        tikz+={
          \pic at ([xshift=\Size].west) {folder};
        },
        align={l}
      }{},
    },
  },
  pic me set/.code n args=2{%
    \forestset{%
      #1/.style={%
        inner xsep=2\Size,
        pic me={pic {#2}},
      }
    }
  },
  pic me set={directory}{folder},
  pic me set={file}{file},
}

\geometry{
    a4paper,
    left=2cm,
    right=2cm,
    top=1.5cm,
    bottom=1.5cm
}
\usepackage{multicol}

\pgfmathdeclarefunction{gauss}{2}{\pgfmathparse{1/(#2*sqrt(2*pi))*exp(-((x-#1)^2)/(2*#2^2))}}

\DeclareRobustCommand{\bbone}{\text{\usefont{U}{bbold}{m}{n}1}}

\DeclareMathOperator{\EX}{\mathbb{E}}

\newcommand{\icol}[1]{% inline column vector
  \left[\begin{smallmatrix}#1\end{smallmatrix}\right]%
}


\titleclass{\subsubsubsection}{straight}[\subsection]

\newcounter{subsubsubsection}[subsubsection]
\renewcommand\thesubsubsubsection{\thesubsubsection.\arabic{subsubsubsection}}
\renewcommand\theparagraph{\thesubsubsubsection.\arabic{paragraph}} % optional; useful if paragraphs are to be numbered

% Set paragraph indent
\setlength{\parindent}{4ex}

% Set the header of the ToC as "Оглавление"
\addto\captionsenglish{
  \renewcommand{\contentsname}
    {Оглавление}
}

\titleformat{\subsubsubsection}
  {\normalfont\normalsize\bfseries}{\thesubsubsubsection}{1em}{}
\titlespacing*{\subsubsubsection}
{0pt}{3.25ex plus 1ex minus .2ex}{1.5ex plus .2ex}

\makeatletter
\renewcommand\paragraph{\@startsection{paragraph}{5}{\z@}%
  {3.25ex \@plus1ex \@minus.2ex}%
  {-1em}%
  {\normalfont\normalsize\bfseries}}
\renewcommand\subparagraph{\@startsection{subparagraph}{6}{\parindent}%
  {3.25ex \@plus1ex \@minus .2ex}%
  {-1em}%
  {\normalfont\normalsize\bfseries}}
\def\toclevel@subsubsubsection{4}
\def\toclevel@paragraph{5}
\def\toclevel@paragraph{6}
\def\l@subsubsubsection{\@dottedtocline{4}{7em}{4em}}
\def\l@paragraph{\@dottedtocline{5}{10em}{5em}}
\def\l@subparagraph{\@dottedtocline{6}{14em}{6em}}
\makeatother

\setcounter{secnumdepth}{4}
\setcounter{tocdepth}{4}

\hypersetup{
	colorlinks=true,
    linkcolor=blue,
    filecolor=magenta,      
    urlcolor=cyan,
}

\definecolor{codegreen}{rgb}{0,0.6,0}
\definecolor{codegray}{rgb}{0.5,0.5,0.5}
\definecolor{codepurple}{rgb}{0.58,0,0.82}
\definecolor{backcolour}{rgb}{0.95,0.95,0.92}

\lstdefinestyle{mystyle}{
    backgroundcolor=\color{backcolour},   
    commentstyle=\color{codegreen},
    keywordstyle=\color{magenta},
    numberstyle=\tiny\color{codegray},
    stringstyle=\color{codepurple},
    basicstyle=\ttfamily\footnotesize,
    breakatwhitespace=false,         
    breaklines=true,                 
    captionpos=b,                    
    keepspaces=true,                 
    numbers=left,                    
    numbersep=5pt,                  
    showspaces=false,                
    showstringspaces=false,
    showtabs=false,                  
    tabsize=2
}

\pagenumbering{arabic}

\begin{document}

    \begin{titlepage}
    \center % Center everything on the page
    \newcommand{\HRline}[1]{\rule{\linewidth}{#1}} % Defines a new command for the horizontal lines, change thickness here
    \vspace*{\fill}
    \textbf{\LARGE Инструкция по взаимодействию с \\ \href{https://github.com/JuiceFV/Emscripten_OpenGL}{Emscripten OpenGL}}
    \vspace*{\fill}
    \begin{multicols}{2}
        \begin{flushleft}
            \large Автор: \\
            \large Касьян Александр Иваночвич
        \end{flushleft}

        
        \columnbreak
        \begin{flushright}
            \large Дата:\\
            \large {\large \today}
        \end{flushright}
    \end{multicols}
    \end{titlepage}

    \newpage

    \tableofcontents

    \newpage
    \section{Вводное слово}
    В данной секции я бы хотел обозначить некоторые моменты, которые могут
    сбивать.
    \begin{itemize}
        \item Данный, как и все остальные документы я оформляю с помощью 
        LaTex, в связи с этим ни doc ни docx файлы предоставить не могу 
        (если таковые потребуются). Все source-файлы находятся на 
        \href{https://github.com/JuiceFV/Emscripten_OpenGL}{GitHub'е}
        в соответствующей папке \emph{docs}
        \item Я заранее прошу прощения, но у меня есть некоторые проблеммы
        с русским языком, в связи с этим данный и все последующие файлы 
        могут быть (скорее всего будут) написаны не очень грамотно и с 
        большим использованием английских слов. 
        \item Данный документ будет пополняться паралельно с ходом выполнения
        проекта.
    \end{itemize}
    \newpage
    \section{Билд}
    Я сразу хочу подметить, что проект не создавался как кроссплатформенное
    решение, т.к. я считаю, что это глупо использовать для кроссплатформы
    OpenGL, т.к. у каждой платформы есть более подходящие спецификации, с 
    более комплексным и гибким API. А следовательно проект работает на
    платформе Windows (а точнее, компилятор MSVC) и Windows/*nix 
    с компилятором emcc (\href{https://emscripten.org/}{Emscripten compiler}).
    За поведение при использовании других компиляторов я ответственности не 
    несу.
    \subsection{Пре-требования}
    \label{sec:pre_req}
    \paragraph{Сборка и запуск с помощью Docker}
    \begin{enumerate}
        \item Если вы хотите самый лёгкий и быстрый запуск, тогда лучше
        просто иметь \href{https://www.docker.com/}{Docker}
        \item Если вы по какой-то причине не установили Docker. Установите
        Docker.
    \end{enumerate}
    \paragraph{Сборка и запуск без Docker}
    \begin{enumerate}
        \item Если вы решили пропустить первые два пункта, тогда потребуются
        \href{https://cmake.org/}{CMake}.
        \item Так же нужно установить \href{https://emscripten.org/}{Emscripten compiler}
        \item Необходимо иметь \href{https://www.python.org/}{Python3} 
        \item Если вы пользователь Windows - лучше иметь 
        \href{https://docs.microsoft.com/ru-ru/windows/wsl/install-win10}{WSL}
        или воспользуйтесь \href{https://docs.microsoft.com/ru-ru/cpp/build/reference/nmake-reference?view=vs-2019}{nmake}
    \end{enumerate}
    \subsection{Установка}
    Варианта установки будет два, первый - я скину zip, второй - с помощью git.
    Лучше пользоваться git, т.к. я использую \emph{git submodule}, а 
    это как-то более нативно, что-ли. Здесь я рассмотрю чисто git.

    \begin{enumerate}
        \item \texttt{> git clone ---recursive https://github.com/JuiceFV/Emscripten\_OpenGL.git}
        \item \texttt{> cd Emscripten\_OpenGL}
        \item Сейчас, если вы в cmd наберёте \colorbox{gray!25!}{\texttt{tree .}}
        то получите более развёрнутую версию этого дерева (здесь приведено всё
        самое нужное для сборки):

        \begin{forest}
          pic dir tree,
          pic root,
          for tree={% folder icons by default; override using file for file icons
            directory,
          },
          [Emscripten\_OpenGL
            [application
            ]
            [cmake
            ]
            [config
            ]
            [doc
            ]
            [CMakeLists.txt, file
            ]
            [Dockerfile, file
            ]
            [docker-compose.yaml, file
            ]
            [requirements.txt, file
            ]
          ]
        \end{forest}
        \item Далее выбирите нужный вам способ. (Да я всё еще деклссирую не Докер вариант)
        \begin{tcolorbox}[colback=green!10!white,colframe=green!70!black,title=\textbf{Docker}]
          \begin{enumerate}
            \item Запустите Докер:
            \begin{itemize}
              \item \textbf{Linux:} \texttt{sudo service docker start}
              \item \textbf{Windows:} \texttt{Win + S $\rightarrow$ "Docker Desktop"}
              $\rightarrow$ Ждём пока значёк кита на панели задач не стабилизируется
              $\rightarrow$ запускаем cmd.
            \end{itemize}
            \item Настройте \hyperref[sec:config_file]{файлы конфигурации}
            \item Находясь в корневой дериктории проекта.
            Разверните сервис командой \colorbox{gray!25!}{\texttt{docker-compose up application}}
            \item Как только консоль зависнет на строчке
            \colorbox{gray!25!}{Runing the server. Follow the link http://<localhost>:<port>}.
            localhost и port устанавливается мануально в файле конфигурации. По дефолту 
            localhost=localhost и port=8080. Перейдите по этой ссылке.
          \end{enumerate}
        \end{tcolorbox}
        \begin{tcolorbox}[colback=yellow!10!white,colframe=red!75!black,title=\textbf{Not Docker}]
          \begin{enumerate}
            \item А может всё таки докер?
            \item (a)????!!!! Пожааааааалуйста!!!!!
            \item Ладно, тогда небольшое разъяснение. 
            Я не заморачивался на тему запуска emcc через Visual Studio.
            так что здесь тоже появляется две опции. 1 - Запуск кода через
            VS без веб ассемблирования. 2 - запуск через emcc с веб ассемблированием.
            \paragraph{Компиляция с помощью EMCC. Желательно делать через WSL}
            Перед сборкой у вас ОБЯЗАТЕЛЬНО должен быть установлен \href{https://emscripten.org/docs/getting_started/downloads.html}{emsdk}.
            А вообще, ещё раз ознакомтесь с \hyperref[sec:pre_req]{pre-requirements}
            \begin{enumerate}
              \item Создайте новую папку в корневой дериктории проекта
              \colorbox{gray!25!}{\texttt{mkdir build}} и перейдите в неё 
              \colorbox{gray!25!}{\texttt{cd build}}
              \item Собираем проект. У emsdk есть возможность собирать с помощью
              конфигурационных файлов, таких как CMake. Собираем:
              \colorbox{gray!25!}{\texttt{emcmake cmake -DFOR\_EMSDK=ON ..}}
              \item На WSL должен появится Makefile. Его тоже нужно собрать:
              \colorbox{gray!25!}{\texttt{emmake make}}
              \item Если вы всё же не получили Makefile, т.к. вы не используете
              WSL, тогда перейдите по \href{https://stackoverflow.com/questions/21387504/cmake-and-make-in-windows}{ссылке}, 
              чтобы собрать и запустить проект без WSL. 
              \item Все исполняемые файлы сохранаяются в \textbf{application/out}
              \item запускаем сервер: 
              \colorbox{gray!25!}{\texttt{python -m http.server}}, переходим
              по ссылке \colorbox{gray!25!}{\texttt{http://<localhost>:<port>}} и открываем
              Emscripten\_Graphics.html
            \end{enumerate}  
          \end{enumerate}
        \end{tcolorbox}
        \begin{tcolorbox}[colback=yellow!10!white,colframe=red!75!black]
        \paragraph{Компиляция без EMCC}
          \begin{enumerate}
            \item 
            \begin{enumerate}
              \item Создайте новую папку в корневой дериктории проекта
              \colorbox{gray!25!}{\texttt{mkdir build}} и перейдите в неё 
              \colorbox{gray!25!}{\texttt{cd build}}
              \item Собираем проект.
              \colorbox{gray!25!}{\texttt{cmake ..}}
              \item Открываем Emscripten\_Graphics.sln и запускаем
            \end{enumerate}
          \end{enumerate}
        \end{tcolorbox}
    \end{enumerate}
  \subsection{Файл конфигурации}
  \label{sec:config_file}
  У меня есть два файла конфигурации, один - для конфигурации сервера другой 
  для конфигурации самого приложения.
  \subsubsection{Конфигурация приложения}
  Данный файл называется appconfig.yaml. Менять название нельзя. Либо в 
  коде нужно напрямую указывать путь к файлу.\\[0.3cm]
  \begin{minipage}{\textwidth}
    \begin{multicols}{2}
      \begin{tcolorbox}[colback=gray!10!white,colframe=black!75!]
        \begin{lstlisting}[language=yaml]
application:
    window:
        width: 640
        height: 480
        \end{lstlisting}
    \end{tcolorbox}
    \columnbreak
    \begin{itemize}
      \item \texttt{window} - параметры окна
      \item \texttt{window:width} - ширина окна 
      \item \texttt{window:height} - высота окна 
    \end{itemize}
    \end{multicols}

\end{minipage}

\end{document}
