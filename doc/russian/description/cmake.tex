\documentclass[a4paper, landscape, 12pt]{article}
\usepackage[T2A]{fontenc}
\usepackage[russian, english]{babel}
\usepackage[utf8x]{inputenc}
\usepackage{amsmath, amssymb}
\usepackage{graphicx}
\usepackage{physics}
\usepackage{float}
\usepackage{booktabs} 
\usepackage{amsmath}
\usepackage{systeme}
\usepackage{blindtext}
\usepackage{enumitem}
\usepackage{listings}
\usepackage[table]{xcolor}
\usepackage{hyperref}
\usepackage{tikz}
\usepackage{titlesec}
\usepackage{mathtools}
\usepackage{tcolorbox}
\usepackage{bm}
\usepackage[normalem]{ulem}
\usepackage{geometry}
\usepackage{siunitx}
\usepackage{xcolor}
\usepackage{tikzsymbols}
\usepackage{relsize}
\usepackage{pgfplots}
\usepackage[edges]{forest}
\usepackage{minted}
\usepackage{csquotes}
\usepackage{pgfplots}

\pgfplotsset{compat=newest}

%========================================
% Define folders and files for dir-tree depict
\definecolor{folderbg}{RGB}{124,166,198}
\definecolor{folderborder}{RGB}{110,144,169}
\newlength\Size
\setlength\Size{4pt}

\tikzset{%
  folder/.pic={%
    \filldraw [draw=folderborder, top color=folderbg!50, bottom color=folderbg] (-1.05*\Size,0.2\Size+5pt) rectangle ++(.75*\Size,-0.2\Size-5pt);
    \filldraw [draw=folderborder, top color=folderbg!50, bottom color=folderbg] (-1.15*\Size,-\Size) rectangle (1.15*\Size,\Size);
  },
  file/.pic={%
    \filldraw [draw=folderborder, top color=folderbg!5, bottom color=folderbg!10] (-\Size,.4*\Size+5pt) coordinate (a) |- (\Size,-1.2*\Size) coordinate (b) -- ++(0,1.6*\Size) coordinate (c) -- ++(-5pt,5pt) coordinate (d) -- cycle (d) |- (c) ;
  },
}
\forestset{%
  declare autowrapped toks={pic me}{},
  declare boolean register={pic root},
  pic root=0,
  pic dir tree/.style={%
    for tree={%
      folder,
      font=\ttfamily,
      grow'=0,
    },
    before typesetting nodes={%
      for tree={%
        edge label+/.option={pic me},
      },
      if pic root={
        tikz+={
          \pic at ([xshift=\Size].west) {folder};
        },
        align={l}
      }{},
    },
  },
  pic me set/.code n args=2{%
    \forestset{%
      #1/.style={%
        inner xsep=2\Size,
        pic me={pic {#2}},
      }
    }
  },
  pic me set={directory}{folder},
  pic me set={file}{file},
}
%========================================

\geometry{
    a4paper,
    left=2cm,
    right=2cm,
    top=1.5cm,
    bottom=1.5cm
}
\usepackage{multicol}

\DeclareRobustCommand{\bbone}{\text{\usefont{U}{bbold}{m}{n}1}}

\DeclareMathOperator{\EX}{\mathbb{E}}

\newcommand{\icol}[1]{% inline column vector
  \left[\begin{smallmatrix}#1\end{smallmatrix}\right]%
}


\titleclass{\subsubsubsection}{straight}[\subsection]

\newcounter{subsubsubsection}[subsubsection]
\renewcommand\thesubsubsubsection{\thesubsubsection.\arabic{subsubsubsection}}
\renewcommand\theparagraph{\thesubsubsubsection.\arabic{paragraph}} % optional; useful if paragraphs are to be numbered

% Set paragraph indent
\setlength{\parindent}{4ex}

% Set the header of the ToC as "Оглавление"
\addto\captionsenglish{
  \renewcommand{\contentsname}
    {Оглавление}
}

\titleformat{\subsubsubsection}
  {\normalfont\normalsize\bfseries}{\thesubsubsubsection}{1em}{}
\titlespacing*{\subsubsubsection}
{0pt}{3.25ex plus 1ex minus .2ex}{1.5ex plus .2ex}

\makeatletter
\renewcommand\paragraph{\@startsection{paragraph}{5}{\z@}%
  {3.25ex \@plus1ex \@minus.2ex}%
  {-1em}%
  {\normalfont\normalsize\bfseries}}
\renewcommand\subparagraph{\@startsection{subparagraph}{6}{\parindent}%
  {3.25ex \@plus1ex \@minus .2ex}%
  {-1em}%
  {\normalfont\normalsize\bfseries}}
\def\toclevel@subsubsubsection{4}
\def\toclevel@paragraph{5}
\def\toclevel@paragraph{6}
\def\l@subsubsubsection{\@dottedtocline{4}{7em}{4em}}
\def\l@paragraph{\@dottedtocline{5}{10em}{5em}}
\def\l@subparagraph{\@dottedtocline{6}{14em}{6em}}
\makeatother

\setcounter{secnumdepth}{4}
\setcounter{tocdepth}{4}

\hypersetup{
	colorlinks=true,
    linkcolor=blue,
    filecolor=magenta,      
    urlcolor=cyan,
}

\definecolor{codegreen}{rgb}{0,0.6,0}
\definecolor{codegray}{rgb}{0.5,0.5,0.5}
\definecolor{codepurple}{rgb}{0.58,0,0.82}
\definecolor{backcolour}{rgb}{0.95,0.95,0.92}

\lstdefinestyle{mystyle}{
    backgroundcolor=\color{backcolour},   
    commentstyle=\color{codegreen},
    keywordstyle=\color{magenta},
    numberstyle=\tiny\color{codegray},
    stringstyle=\color{codepurple},
    basicstyle=\ttfamily\footnotesize,
    breakatwhitespace=false,         
    breaklines=true,                 
    captionpos=b,                    
    keepspaces=true,                 
    numbers=left,                    
    numbersep=5pt,                  
    showspaces=false,                
    showstringspaces=false,
    showtabs=false,                  
    tabsize=2
}

\pagenumbering{arabic}

\begin{document}
    \begin{titlepage}
        \center % Center everything on the page
        \newcommand{\HRline}[1]{\rule{\linewidth}{#1}} % Defines a new command for the horizontal lines, change thickness here
        \vspace*{\fill}
        \textbf{\LARGE Описание CMake инструкций}
        \vspace*{\fill}
        \begin{multicols}{2}
            \begin{flushleft}
                \large Автор: \\
                \large Касьян Александр Иваночвич
            \end{flushleft}

            
            \columnbreak
            \begin{flushright}
                \large Дата:\\
                \large {\large \today}
            \end{flushright}
        \end{multicols}
    \end{titlepage}
    \newpage
    \tableofcontents
    \newpage
    \section{Вводное слово}
    \noindent
    \href{https://en.wikipedia.org/wiki/CMake}{CMake} является одним из главных, если не главным инструментом в сборке
    даного проекта. CMake - это система автоматизации сборки из исходников.
    Сам CMake не собирает проект, он генерирует файлы для управления сборкой.
    В данном файле я разбиру инструкции, которые я написал для сборки 
    \textbf{Emscripten\_OpenGL}. По ходу проекта данный файл будет изменться
    все логи можно будет увидеть \hyperref[sec:fickle]{здесь}
    \newpage
    \section{Изменения}
    \label{sec:fickle}
    \newpage
    \section{Основной CMakeList.txt}
    \inputminted{cmake}{../../../CMakeLists.txt}
    \newpage
    \noindent
    Для начала output-дириктории:
    \begin{minted}
        [
            frame=lines,
            framesep=2mm,
            baselinestretch=1.2,
            bgcolor=lightgray!20!,
            fontsize=\footnotesize,
        ]{cmake}
set(CMAKE_ARCHIVE_OUTPUT_DIRECTORY ${CMAKE_BINARY_DIR}/lib)
set(CMAKE_LIBRARY_OUTPUT_DIRECTORY ${CMAKE_BINARY_DIR}/lib)
set(CMAKE_RUNTIME_OUTPUT_DIRECTORY ${CMAKE_BINARY_DIR}/bin)
    \end{minted}
    Все рантайм объекты будут находится по пути \colorbox{gray!25!}{\texttt{bin/<Configuration>/}},
    где Configuration - это конфигурация, такая как Debug/Release. Все статические 
    библиотеки будут храниться в  \colorbox{gray!25!}{\texttt{lib/<Configuration>/}}.
\end{document}